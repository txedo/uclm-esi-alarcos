En esta iteraci�n (Tabla \ref{table:ficha-iteracion-9}) se aborda:
\begin{itemize}
	\item Implementaci�n de las funcionalidades de gesti�n de compa��as, factor�as, proyectos y subproyectos.
	\item Pruebas unitarias y funcionales de las funcionalidades implementadas en esta misma iteraci�n.
\end{itemize}


\subsubsection{Disciplina de implementaci�n}
\label{sec:iteracion-9-implementacion}

Durante esta disciplina se implementan las funcionalidades relativas a la gesti�n de compa��as, factor�as, proyectos y subproyectos, y se corrigen algunos bugs que se han detectado en el sistema. Adem�s se obtienen los diagramas de componentes y de despliegue correspondientes.

En la figura \ref{fig:it9-factory-form-screenshot} se muestra un fragmento de la p�gina que permite consultar los datos de una factor�a de software, as� como realizar operaciones con los proyectos que lidera y los subproyectos que est� desarrollando.

\imagen{Cap5//it9-factory-form-screenshot}{1.5}{Captura de pantalla de un fragmento de la p�gina que permite consultar informaci�n de una factor�a de software}{fig:it9-factory-form-screenshot}




\subsubsection{Disciplina de pruebas}
\label{sec:iteracion-9-pruebas}

Durante esta disciplina se construyen los casos de prueba que sirven para
comprobar la funcionalidad de los requisitos implementados la disciplina de implementaci�n de esta misma iteraci�n.

Al tratarse de funcionalidades t�picas de gesti�n de informaci�n no se requiere un estudio m�s detallado en este documento.