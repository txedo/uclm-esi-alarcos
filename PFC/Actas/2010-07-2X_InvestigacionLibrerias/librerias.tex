\section{Librer�as}

\subsection{Desarrollo Web}

\subsubsection{DWR: Direct Web Remoting}

\noindent \textbf{Versi�n:} 2.0 stable \\
\noindent \textbf{Licencia:} Apache Software License v2 \cite{license:apache-v2} \\
\noindent \textbf{P�gina Web:} \cite{homepage:dwr}

Easy Ajax for Java



\subsubsection{JQuery}

\noindent \textbf{Versi�n:} \\
\noindent \textbf{Licencia:} \\
\noindent \textbf{P�gina Web:} 

Descripci�n
% http://jquery.com/
% http://es.wikipedia.org/wiki/JQuery



\subsubsection{JQuery UI}

\noindent \textbf{Versi�n:} \\
\noindent \textbf{Licencia:} \\
\noindent \textbf{P�gina Web:} 

Descripci�n
% http://jqueryui.com/
% http://es.wikipedia.org/wiki/JQuery_UI


\subsubsection{YUI: Yahoo! UI Library}

\noindent \textbf{Versi�n:} \\
\noindent \textbf{Licencia:} BSD License \\
\noindent \textbf{P�gina Web:} 

Descripci�n
% http://developer.yahoo.com/yui/
% http://es.wikipedia.org/wiki/Yahoo_User_Interface



\subsubsection{GWT: Google Web Toolkit}

\noindent \textbf{Versi�n: 2.0} \\
\noindent \textbf{Licencia:} Apache Software License v2 \cite{license:apache-v2} \\
\noindent \textbf{P�gina Web:} 

Descripci�n
% http://code.google.com/intl/es-ES/webtoolkit/
% http://es.wikipedia.org/wiki/Google_Web_Toolkit





\subsection{Visualizaci�n}

Se han analizado numerosas librer�as de visualizaci�n tales como jQuery Visualize \cite{homepage:jquery-visualize}, Flot \cite{homepage:flot}, Axiis \cite{homepage:axiis}, Style Chart \cite{homepage:style-chart}, JFree Chart \cite{homepage:jfree-chart} y Google Chart API \cite{homepage:google-chart-api}, entre otros. Finalmente se han seleccionado para profundizar en su estudio las que se enumeran a continuaci�n. El criterio de selecci�n se fundamenta en las siguientes caracter�sticas:
\begin{itemize}
	\item Generaci�n din�mica de diagramas.
	\item Posibilidad de elecci�n entre una gran colecci�n de modelos (diagramas de barras, sectores, etc.).
	\item Permitir la interacci�n por parte del usuario.
	\item Visualizaci�n atractiva.
	\item F�cil integraci�n en la aplicaci�n Web, desarrollada en Java.
	\item Gratis.
	\item Una licencia conforme a nuestros requisitos.
\end{itemize}


\subsubsection{Google Visualization API}

\noindent \textbf{Versi�n} \\
\noindent \textbf{Licencia:} \\
\noindent \textbf{P�gina Web: \cite{homepage:google-visualization-api}} 

Descripci�n



\subsubsection{Protovis}

\noindent \textbf{Versi�n} \\
\noindent \textbf{Licencia:} \\
\noindent \textbf{P�gina Web:} 

Descripci�n



\subsubsection{The javaScript InfoVis Toolkit}

\noindent \textbf{Versi�n} \\
\noindent \textbf{Licencia:} \\
\noindent \textbf{P�gina Web:} 

Descripci�n



\subsubsection{Open Flash Chart}

\noindent \textbf{Versi�n} \\
\noindent \textbf{Licencia:} \\
\noindent \textbf{P�gina Web: \cite{homepage:open-flash-chart}} 

Descripci�n



\subsubsection{JOGL: JavaOpenGL}

\noindent \textbf{Versi�n} \\
\noindent \textbf{Licencia:} \\
\noindent \textbf{P�gina Web:} 

Descripci�n



\subsubsection{WebGL: OpenGL ES 2.0 for the Web} \label{sec:webgl}

\noindent \textbf{Versi�n} \\
\noindent \textbf{Licencia:} \\
\noindent \textbf{P�gina Web:} 

Descripci�n
% http://www.khronos.org/webgl/
% http://en.wikipedia.org/wiki/WebGL
% http://khronos.org/webgl/wiki/Main_Page
% http://www.khronos.org/webgl/wiki/Getting_a_WebGL_Implementation