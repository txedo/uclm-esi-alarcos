\section{Introducci�n}

El sistema a desarrollar se puede categorizar como una \textbf{Aplicaci�n de Internet Enriquecida}, en ingl�s \textbf{R}ich \textbf{I}nternet \textbf{A}pplication o \textbf{RIA}. Una RIA, concepto cada vez m�s com�n en la era de la Web 2.0, \cite{wikipedia:ria} es una aplicaci�n Web que, siendo accedida a trav�s de un navegador Web est�ndar, trata de satisfacer las necesidades del usuario como si de una aplicaci�n de escritorio se tratase, combinando as� las ventajas de ambos paradigmas (aplicaci�n Web y aplicaci�n de escritorio) para maximizar la experiencia del mismo.

Una RIA se difiere de una p�gina Web tradicional en el sentido de que no hay una recarga continua de p�ginas cada vez que el usuario pulsa sobre un enlace. Esto evita producir un alto tr�fico de informaci�n entre el cliente y el servidor por grande o peque�a que sea la informaci�n que se desea actualizar. Adem�s, una RIA posee una gran capacidad multimedia ya que es posible empotrar distintos tipos de objetos (applets, videos, sonidos) en ella sin necesidad de utilizar programas externos para su visualizaci�n.

Una vez establecidas las caracter�sticas que debe cumplir una RIA y estudiadas las capacidades de visualizaci�n que posee Sonar \cite{homepage:sonar}, se proponen un conjunto de tecnolog�as, librer�as y servicios que permitir�n implementar un sistema que satisfaga los requisitos de usuario correspondientes.