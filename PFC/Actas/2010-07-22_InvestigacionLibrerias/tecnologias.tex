\section{Tecnolog�as}

La capa de conocimiento o reglas de negocio del sistema estar�n desarrolladas en Java, por lo que se propone JSP (Java Server Pages) como tecnolog�a principal para el desarrollo de la aplicaci�n web. Se har� uso de otras tecnolog�as comunes entre la Web 2.0 como CSS, JavaScript y Ajax. No obstante, con un prop�sito investigador, se propone tambi�n el uso de HTML 5, Struts 2 e Hibernate.



\subsection{HTML 5}

HTML 5 (HyperText Markup Language, versi�n 5) \cite{wikipedia:html5} es la quinta revisi�n de HTML.  Esta nueva versi�n incorpora de forma nativa, es decir, sin depender de programas externos o \textit{plugins} del navegador, reproducci�n de audio y v�deo, edici�n de documentros, \textit{drag\&drop}, un elemento \textit{canvas} sobre el que se pueden pintar gr�ficos en 2 y 3 dimensiones, etc. Esta �ltima caracter�stica hace que sea posible el uso de WebGL (ver secci�n \ref{sec:webgl}).



\subsection{Struts 2}

Struts 2 es la nueva versi�n del \textit{framework} de desarrollo web Java Apache Struts. Struts 2 est� basado en el patr�n MVC (Modelo-Vista-Controlador), una arquitectura que busca reducir el acoplamiento dividiendo las responsabilidades en 3 capas bien diferenciadas:

\begin{itemize}
	\item El modelo, que hace referencia a los datos que maneja la aplicaci�n y las reglas de negocio que operan sobre ellos.
	\item La vista, encargada de generar la interfaz con la que la aplicaci�n interacciona con el usuario.
	\item El controlador, que comunica la vista y el modelo respondiendo a eventos generados por el usuario en la vista, invocando cambios en el modelo, y devolviendo a la vista la informaci�n del modelo necesaria para que pueda generar la respuesta adecuada para el usuario.
\end{itemize}
    
No obstante, tambi�n es posible dise�ar una aplicaci�n web mediante el patr�n MVC utilizando p�ginas JSP y servlets. En este caso, las p�ginas JSP tomar�an el rol de la vista, los servlets el del controlador y los POJO (Plan Old Java Object) el del modelo. El modelo, a su vez, suele estar dividido en dos subcapas siguiendo el patr�n DAO (Data Access Object).

En la aplicaci�n web, Struts 2 se emplear� en el desarrollo de determinados casos de uso extra�dos de la especificaci�n de requisitos de usuario, identificando dichos casos de uso como \textit{acciones} de Struts.

En \cite{homepage:struts2-plugin-registry} se encuentra una lista con todos los plugins de los que dispone este framework: jQuery, Hibernate, jFree Chart, JUnit, YUI, GWT, etc.



\subsection{Hibernate}

Hibernate es un \textit{framework} de persistencia, libre y bajo licencia LGPL, que permite establecer un \textit{mapeo objeto-relacional} para la plataforma Java y sistemas basados en SQL, es decir, permite relacionar una clase Java con una tabla de una base de datos para abstraer al programador del lenguaje SQL y haciendo que la aplicaci�n sea portable entre cualquier sistema gestor de bases de datos compatible con SQL.

Este framework, adem�s de utilizarse para gestionar la persistencia del n�cleo de Medusas (ver Figura \ref{fig:medusas-arch}), se utilizar� para gestionar la persistencia de la aplicaci�n web: gesti�n de usuarios, estad�sticas de uso, historiales, etc.

\imagen{./images/medusas-arch.png}{0.7}{Arquitectura del proyecto Medusas}{fig:medusas-arch}