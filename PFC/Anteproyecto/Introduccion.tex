\section{Introducci�n}

En los �ltimos a�os, las organizaciones y empresas se han centrado en la calidad de los procesos que se siguen para desarrollar software, pero dados los problemas que han surgido en problemas ya implantados y en producci�n, ha sido necesario actualizar la percepci�n que se tiene acerca de la calidad del software. Por ello, las organizaciones han comenzado a preocuparse por la calidad del producto software desde el punto de vista de los clientes y usuarios finales con el fin de aumentar el grado de satisfacci�n y confianza por parte de los mismos. Para cubrir este aspecto se ha estado trabajando en las pruebas (\textit{testing}) con el fin de asegurar una correcta funcionalidad y usabilidad del sistema, pero se han quedado sin cubrir aspectos de seguridad y mantenibilidad. Y es que un producto software no es �nicamente el c�digo que se ejecuta sobre una m�quina, si no que lo son tambi�n los artefactos que se han ido produciendo a lo largo de todo el ciclo de vida: modelos, diagramas, documentos, etc.

Con dichos requisitos, se comenzaron a utilizar dos est�ndares:
\begin{myitemize}
	\item \textbf{ISO/IEC 9126}\cite{iso-9126}, que define el modelo de calidad de un producto, incluyendo las caracter�sticas, subcaracter�sticas y m�tricas (de calidad interna, calidad externa y calidad en uso) que han de tenerse en cuenta, pero no proporciona una metodolog�a para su evaluaci�n. Las caracter�sticas que propone son funcionalidad, fiabilidad, usabilidad, eficiencia, mantenibilidad y portabilidad.
	\item \textbf{ISO/IEC 14598}\cite{iso-14598}, que proporciona gu�as y requisitos para el proceso de evaluaci�n, cubriendo as� los aspectos de los que carece la norma ISO/IEC 9126.
\end{myitemize}

Pero distintas incompatibilidades e inconsistencias entre ambos est�ndares, tales como diferencias en el vocabulario y t�rminos, o la necesidad de a�adir recomendaciones, metodolog�as y gu�as, as� como nuevas necesidades para especificar las dimensiones de la calidad del software, motivan la creaci�n de una nueva familia denominada SQuaRe (\underline{S}oftware product \underline{Qua}lity \underline{R}equirements and \underline{E}valuation)