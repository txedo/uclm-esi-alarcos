\section{M�todo y fases de trabajo}

Debido al car�cter relativamente investigador que posee este proyecto, en el que se producir�n numerosas adiciones y modificaciones de requisitos de usuario a lo largo de su ciclo de vida, se ha optado por utilizar una metodolog�a de trabajo gen�rica que permita adaptarse a este caso de estudio. Por ello, la metodolog�a elegida es el \textbf{Proceso Unificado de Desarrollo} (en adelante PUD).

\subsection{El Proceso Unificado de Desarrollo}

El PUD \cite{pud} es una evoluci�n del Proceso Unificado de Rational, que define un ``conjunto de actividades necesarias para transformar los requisitos de usuario en un sistema software''. Es un marco gen�rico que puede especializarse para una gran variedad de sistemas de software, para diferentes �reas de aplicaci�n, diferentes tipos de organizaciones, diferentes niveles de aptitud y diferentes tama�os de proyectos. Sus principales caracter�sticas son:
\begin{myitemize}
	\item \textbf{Dirigido por casos de uso}. Para poder desarrollar un sistema es necesario saber qu� necesitan sus usuarios. Un usuario, en semejanza al concepto de \textit{actor} en UML, puede ser un ser humano u otro sistema que interacciona con el sistema que se est� desarrollando. Las necesidades de un usuario se denominan requisitos funcionales y se representan por medio de casos de uso. Por tanto, un caso de uso representa un requisito funcional del sistema que proporciona un resultado de valor a un usuario. Adem�s, gu�an el proceso de desarrollo desde la especificaci�n de requisitos hasta las pruebas y sirven a los desarrolladores para crear una serie de modelos que les permita llevarlos a cabo. Todos los casos de uso juntos constituyen el \textbf{modelo de casos de uso}.
	\item \textbf{Centrado en la arquitectura}. Un sistema software puede contemplarse desde varios puntos de vista. Por tanto, la arquitectura software incluye los aspectos est�ticos y din�micos m�s significativos del sistema y debe estar profundamente relacionada con los casos de uso ya que debe permitir el desarrollo de los mismo. Tanto la arquitectura como los casos de uso deben desarrollarse en paralelo.
	\item \textbf{Iterativo e incremental}. Partiendo del dicho latino \textit{divide et impera} (en castellano, \textit{divide y vencer�s}), la filosof�a del PUD se basa en la creaci�n de \textbf{mini-proyectos} para repartir el trabajo. Cada mini-proyecto es una \textbf{iteraci�n} que resulta en un \textbf{incremento}.
	
	Cada mini-proyecto sigue el esquema \textit{an�lisis, dise�o, implementaci�n y pruebas}. Cada iteraci�n trata un grupo de casos de uso que juntos ampl�an la utilidad del producto desarrollado y trata los riesgos m�s importantes. Si la iteraci�n cumple sus objetivos, se contin�a con la pr�xima.
\end{myitemize}


\subsubsection{Ciclo de Vida del PUD}

Debido al enfoque iterativo e incremental que caracteriza al PUD, �ste se divide en ciclos que a su vez se dividen en fases.