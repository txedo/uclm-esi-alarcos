\section{Objetivos}

Los objetivos que se esperan lograr con la realizaci�n de este PFC se han dividido en dos categor�as. En la primera de ellas, como objetivo principal, se introducir� el producto final que se espera obtener y, por �ltimo, como objetivos secundarios, se expondr�n los objetivos que se ir�n alcanzando en las distintas fases de elaboraci�n del PFC.


\subsection{Objetivo principal}

El PFC consiste en la elaboraci�n de una \textbf{aplicaci�n web} para la visualizaci�n de los datos obtenidos como resultado de las operaciones de medici�n y evaluaci�n de la calidad del producto software del entorno MEDUSAS, que est� basado en la nueva familia de normas ISO/IEC 25000. Para ello, se desarrollar� un proyecto Java que ser� capaz de representar gr�ficamente los distintos niveles de detalle procedentes de los resultados de evaluaci�n de la calidad del producto software.

Dicha aplicaci�n web ser� desarrollada utilizando las �ltimas tecnolog�as empleadas en desarrollo y dise�o web (Ver Secci�n \ref{sec:medios}) con el fin de obtener una RIA (Rich Internet Application) \cite{def:ria}.

Una RIA se fundamenta en una arquitectura cliente/servidor as�ncrona, segura y escalable, disminuyendo la carga de trabajo en el servidor y siendo accesible desde cualquier navegador web. As�, estar� disponible para cualquier persona desde cualquier rinc�n del planeta, tratando de maximizar la experiencia de usuario y producir un incremento en la productividad.


\subsection{Objetivos secundarios} \label{sec:objetivos-secundarios}

A lo largo del proceso de investigaci�n y elaboraci�n del PFC se alcanzar�n las siguientes metas:
\begin{myitemize}
	\item Comprender las actuales normas sobre la calidad del software.
	\item Investigar sobre las principales formas de visualizaci�n de calidad del software.
	\item Estudiar los principales entornos y medios actuales para la visualizaci�n de la calidad del software, as� como las librer�as gr�ficas que permiten su implementaci�n.
	\item Implementar un entorno que permita visualizar la calidad de un producto software de acuerdo a las caracter�sticas de calidad impuestas por la ISO/IEC 25000.
	\item Integrar dicho entorno de visualizaci�n con un entorno real de medici�n de la calidad del producto software.
\end{myitemize}