\section{Medios que se pretenden utilizar} \label{sec:medios}

Los medios y recursos necesarios para el desarrollo y despliegue del PFC se categorizan en \textit{medios hardware} y \textit{medios software}.


\subsection{Medios hardware}

Tanto para el desarrollo como el despliegue del sistema bastar� con un ordenador sobremesa o port�til. Dicho ordenador deber� estar dotado de una conexi�n a Internet y actuar como servidor web para permitir que los usuarios puedan acceder a la aplicaci�n por medio de un navegador web convencional.

La tarjeta gr�fica del sistema requiere especial menci�n. �sta no tiene por qu� poseer una potencia extraordinaria, pero es de vital importancia que disponga de sus controladores gr�ficos debidamente instalados para una correcta ejecuci�n del motor gr�fico.


\subsection{Medios software}

Tanto la aplicaci�n web como el motor gr�fico ser�n desarrollados en tecnolog�as basadas en Java y, debido a su naturaleza \textit{multiplataforma}, se espera que no existan restricciones en cuanto al sistema operativo que el ordenador debe ejecutar. Sin embargo, �ste debe tener debidamente instalado el entorno de ejecuci�n Java (JRE o \textit{Java Runtime Environment})

Por otro lado, son necesarias las siguientes herramientas:
\begin{myitemize}
	\item JDK 1.4+ (\textit{Java Development Kit}).
	\item Un entorno integrado de desarrollo (IDE o \textit{Integrated Development Environment}). Se utilizar� \textbf{Eclipse EE Helios}.
	\item Un navegador web con JavaScript habilitado para que se pueda establecer una comunicaci�n entre la aplicaci�n web y el motor gr�fico, que estar� incrustado en un applet, y poder beneficiarse de las caracter�sticas de la tecnolog�a AJAX (\textit{Asynchronous JavaScript And XML}).
	\item Un sistema para la gesti�n y construcci�n de proyectos. Se har� uso de \textbf{Maven 2}.
	\item Un servidor web capaz de desplegar aplicaciones web con estructura WAR (\textit{Web-Archive} ). \textbf{Apache Tomcat 7} es la soluci�n m�s factible gracias a su licencia, eficiencia y rendimiento.
	\item Un sistema gestor de bases de datos. Se emplear� \textbf{MySQL Community Server 5.x}.
	\item \textbf{Visual Paradigm} para modelado y dise�o de diagramas UML.
	\item \LaTeX{} para la elaboraci�n de documentaci�n t�cnica asociada al proyecto.
	\item GIMP e Inkscape para el dise�o y edici�n de figuras, im�genes y gr�ficos.
\end{myitemize}
