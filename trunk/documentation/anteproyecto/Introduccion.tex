\section{Introducci�n}

La \textbf{globalizaci�n}, seg�n la Real Academia Espa�ola, es la \textit{``tendencia de los mercados y de las empresas a extenderse, alcanzando una dimensi�n mundial que sobrepasa las fronteras nacionales''}. Otras fuentes \cite{croucher04} buscan una definici�n m�s expl�cita y definen este concepto como \textit{``un proceso econ�mico, tecnol�gico, social, pol�tico y cultural de interacci�n e integraci�n entre personas, empresas y gobiernos de todo el mundo''}. No cabe duda de que este proceso o tendencia ha desencadenado una gran cantidad de cambios no s�lo en la forma de trabajo tradicional de la industria y el comercio, si no tambi�n en el modo en el que los productos son dise�ados, construidos, probados y distribuidos \cite{ohara94}.

El fen�meno de la globalizaci�n tambi�n est� afectando a la industria del software. A consecuencia de ello, el software est� siendo desarrollado por personas que no se encuentran co-localizadas en las mismas ubicaciones, es decir, se encuentran distribuidas o dispersas geogr�ficamente. Esta dispersi�n geogr�fica puede limitarse a un �rea cerrada y pr�xima (por ejemplo, distintos edificios en una misma localidad) o expandirse por todo el mundo evadiendo cualquier frontera. Adem�s, los integrantes de un equipo de desarrollo pueden pertenecer a una misma organizaci�n o a organizaciones diferentes \cite{katzy99}. Cuando varios equipos de desarrollo, que trabajan en un mismo proyecto, pertenecen a distintas organizaciones y adem�s est�n dispersos geogr�ficamente por todo el mundo, se habla de \textbf{Desarrollo Global de Software} (en ingl�s, Global Software Development o GSD).

El GSD introduce una serie de caracter�sticas y complejidades que afectan, tanto positiva como negativamente, al desarrollo y gesti�n de proyectos. Por un lado, este paradigma permite a las organizaciones el abstraerse de la distancia geogr�fica y minimizar costes, pero es susceptible de problemas como las diferencias ling��sticas, culturales, horarias, etc�tera. % a�adir referencias

Con el fin de proporcionar una soluci�n que permita a las organizaciones conocer el estado de los proyectos que se est�n desarrollando de forma global, y simplificar las tareas de gesti�n de las personas a cargo de los mismos, se propone este \textbf{Proyecto Fin de Carrera}, en adelante PFC, que consistir� en construir una herramienta enmarcada en un proyecto de investigaci�n que ser� utilizada por \textbf{Indra Sistemas, S.A.}.
