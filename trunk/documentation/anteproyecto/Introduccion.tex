\section{Introducci�n}

La \textbf{globalizaci�n}, seg�n la Real Academia Espa�ola, es la \textit{tendencia de los mercados y de las empresas a extenderse, alcanzando una dimensi�n mundial que sobrepasa las fronteras nacionales}. Otras fuentes \cite{croucher04} describen este concepto como \textit{un proceso econ�mico, tecnol�gico, social, pol�tico y cultural de interacci�n e integraci�n entre personas, empresas y gobiernos de todo el mundo}. Este proceso ha desencadenado una gran cantidad de cambios no s�lo en la forma de trabajo tradicional de la industria y el comercio, si no tambi�n en el modo en el que los productos son dise�ados, construidos, probados y distribuidos a los respectivos clientes \cite{ohara94}. Por lo tanto, la industria del software y las empresas que dependen del software tambi�n se vean afectadas por dicho proceso y sus consecuencias.

Debido al fen�meno de la globalizaci�n, el software est� siendo desarrollado por personas que no se encuentran en las mismas ubicaciones, es decir, se encuentran distribuidas geogr�ficamente. Este desarrollo distribuido afecta tanto al proceso de desarrollo del software como a la gesti�n de proyectos, lo cual supone (((%a�adir algunos problemas y las citas
))) pero tambi�n proporciona una serie de beneficios muy importantes que prometen, entre otros, beneficios econ�micos \cite{herb-moitra-2001}. A continuaci�n se enumeran algunos de ellos:
\begin{myitemize}
	\item \textbf{Reducci�n de costes:} gracias a la subcontrataci�n de empresas o equipos de desarrollo ubicados en localidades o pa�ses en los que la mano de obra es m�s econ�mica.
	\item \textbf{Tiempos de lanzamiento al mercado m�s cortos:} ya que se pueden subcontratar varios equipos de desarrollo para desempe�ar distintas actividades de un modo paralelo o concurrente.
	\item \textbf{Mejora de la calidad:} se debe a que determinadas funcionalidades pueden ser desarrolladas por expertos ajenos a la empresa que desarrolla el software.
	\item \textbf{Escalabilidad:} hace posible afrontar la alza o baja, tanto temporal como permanente, de demanda de productos
\end{myitemize}

