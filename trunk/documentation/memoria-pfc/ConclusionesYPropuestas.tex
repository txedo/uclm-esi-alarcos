\chapter{Conclusiones y Propuestas}
\label{ch:conclusionesypropuestas}

% Breve resumen de lo m�s destacable del PFC con la soluci�n propuesta y posibles mejoras, ampliaciones o trabajos relacionados que quedan por hacer y que tienen inter�s para el tema tratado. Al final, firma del autor.

Dos componentes que dan forma a un sistema de visualizaci�n adaptable, configurable y extensible.

Motor gr�fico completamente independiente de la aplicaci�n que lo contiene, se puede apreciar en el diagrama de paquetes de la figura \ref{fig:conclusiones-package-diagram}. Las �nicas dependencias existen, de manera \textbf{unidireccional} es para leer las dimensiones de las met�foras  y para construir la cadena JSON mediante el objeto City, ya que esa estructura es la que maneja el motor gr�fico como cadena de entrada.

\imagenBorde{Cap5//conclusiones-package-diagram}{0.49}{Diagrama de paquetes de los dos componentes integrados}{fig:conclusiones-package-diagram}