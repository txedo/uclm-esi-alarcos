% Se debe recordar que aunque en el tipo de documento report se puede incluir un abstract (resumen). En la clase book este entorno no est� definido.
%====================================
%====================================
%.... RESUMEN (opcional, 1 p�g.)
\selectlanguage{spanish}%
\begin{abstract}
La globalizaci�n es un \textit{fen�nemo que est� afectando a aspectos econ�micos, tecnol�gicos, sociales, pol�ticos y culturales que se dan entre las personas, empresas y gobiernos de todo el mundo}. Esto provoca que la industria y el comercio, entre otros, rompan con las fronteras nacionales y se expandan a todo el mundo con el objetivo de llegar a un mayor p�blico y reducir costes.

La industria del software no es una excepci�n y tambi�n se est� amoldando a las nuevas modas que se imponen con el paso del tiempo. Por ello, el software est� pasando de ser desarrollado en una �nica ubicaci�n a ser desarrollado en m�ltiples ubicaciones dispersas por todo el mundo. A esto se le conoce como \textit{Desarrollo Global de Software} (DSG).

El DSG aporta una gran cantidad de ventajas a las empresas que desarrollan software bajo este contexto. No obstante, tambi�n introduce una serie de factores que pueden afectar negativamente a aspectos de calidad y productividad, entre otros, si no se controlan adecuadamente. Para ello, son necesarios entornos y herramientas que ayuden a asegurar la calidad del software y a mitigar el efecto de la distribuci�n de recursos en el proceso de desarrollo y en la gesti�n del conocimiento.

En este Proyecto Fin de Carrera se propone desarrollar una herramienta que permitir� visualizar medidas e indicadores de calidad bajo el contexto del desarrollo global. Para ello ser� necesario sintetizar la gran cantidad de informaci�n con la que cuentan los profesionales e investigadores a cargo de este tipo de actividades y proporcionar mecanismos para facilitar la utilizaci�n del conocimiento organizacional en las actividades de desarrollo. Gracias a ello ser� posible predecir y detectar los distintos riesgos que pueden ocurrir, as� como ofrecer soporte a la toma de decisiones en las actividades de la organizaci�n con el objetivo de incrementar su competitividad a nivel internacional.
\end{abstract}

%====================================
%====================================
%.... ABSTRACT (opcional, 1 p�g.)
% Aqu� se muestra un ejemplo completo en el que se a�ade una versi�n en ingl�s del resumen.
\selectlanguage{english}% Selecci�n de idioma ingl�s.
\begin{abstract}
... english version for the abstract ...
Se traducir� cuando la versi�n en espa�ol quede validada.
\end{abstract}

\selectlanguage{spanish}% Para el resto del documento del idioma es espa�ol.
%====================================
%====================================
%.... AGRADECIMIENTOS (opcional, 1 p�g. o m�s, recomendable hacerloe en una s�lo)
% Los agradecimientos son m�s extensos que la dedicatoria como se muestra en el ejemplo.
\chapter*{Agradecimientos} % Opci�n con * para que no aparezca en TOC
Escribir agredecimientos
