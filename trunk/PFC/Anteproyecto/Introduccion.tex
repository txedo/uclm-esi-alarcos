\section{Introducci�n}

A�os atr�s, las organizaciones y empresas se preocupaban de la calidad de los procesos que se segu�an para desarrollar software bas�ndose en la premisa de que \textit{``con un proceso de calidad se obtiene un producto de calidad''} \cite{mpiattini-medusas}, pero dados los problemas que han surgido en sistemas ya implantados y en producci�n, y las p�rdidas que �stos han ocasionado, ha sido necesario actualizar la percepci�n que se tiene acerca de la calidad del software. Por ello, las organizaciones comenzaron a preocuparse por la calidad del producto software desde el punto de vista de los clientes y usuarios finales, abarcando un mayor n�mero de \textit{stakeholders}, con el fin de aumentar el grado de satisfacci�n y confianza por parte de los mismos.

Es por esto por lo que se tratado de asegurar la funcionalidad y usabilidad de los sistemas profundizando en �rea de \textit{testing}, pero se han quedado sin cubrir aspectos de seguridad y mantenibilidad. Y es que un producto software no es �nicamente el c�digo que se ejecuta sobre una m�quina, si no que lo son tambi�n los artefactos que se han ido produciendo a lo largo de todo el ciclo de vida: modelos, diagramas, documentos, etc.

Para paliar algunos de los problemas expuestos anteriormente, se comenzaron a utilizar dos est�ndares:
\begin{myitemize}
	\item \textbf{ISO/IEC 9126}\cite{iso-9126}, que define el modelo de calidad de un producto, incluyendo las caracter�sticas, subcaracter�sticas y m�tricas (de calidad interna, calidad externa y calidad en uso) que han de tenerse en cuenta, pero sin proporcionar una metodolog�a para su evaluaci�n. Las caracter�sticas que propone son funcionalidad, fiabilidad, usabilidad, eficiencia, mantenibilidad y portabilidad.
	\item \textbf{ISO/IEC 14598}\cite{iso-14598}, que proporciona gu�as y requisitos para el proceso de evaluaci�n, cubriendo as� los aspectos de los que carece la norma ISO/IEC 9126.
\end{myitemize}

Pero entre ambos est�ndares existen distintas incompatibilidades e inconsistencias -tales como diferencias en el vocabulario y t�rminos, la necesidad de a�adir recomendaciones, metodolog�as, gu�as y nuevas necesidades para especificar las dimensiones de la calidad del software, entre otros- que motivan la creaci�n de la familia \textbf{ISO/IEC 25000}, denominada SQuaRe (\underline{S}oftware product \underline{Qua}lity \underline{R}equirements and \underline{E}valuation).

La norma ISO/IEC 25000, que consiste en 14 documentos agrupados en 5 vol�menes \cite{square-second-generation}, establece criterios para la especificaci�n de requisitos de calidad de productos software y su evaluaci�n, un modelo de calidad, as� como m�tricas recomendadas para los distintos atributos de dicho modelo, teniendo en cuenta a todos los \textit{stakeholders} (clientes, usuarios finales y factor�as y empresas de desarrollo).

La aparici�n de esta norma, cuyo objetivo es reemplazar a las normas ISO/IEC 9126 e ISO/IEC 14598, junto con las nuevas necesidades que surgen en las empresas para controlar y evaluar la calidad y seguridad de los desarrollos inform�ticos, motivan la aparici�n del proyecto \textbf{MEDUSAS} (\underline{M}ejora y \underline{E}valuaci�n del \underline{D}ise�o, \underline{U}sabilidad, \underline{S}eguridad y \underline{M}antenibilidad del \underline{S}oftware). MEDUSAS (Fig. \ref{fig:arch-medusas}) es un entorno metodol�gico e instrumental basado en la ISO/IEC 25000 para satisfacer dichas necesidades, y evaluar la calidad y seguridad del software teniendo en cuenta todos y cada uno de sus artefactos: c�digo fuente, dise�o, diagramas, modelos, etc.

\imagen{./images/comp_medusas.png}{0.4}{Arquitectura del proyecto MEDUSAS}{fig:arch-medusas}

Aqu� es donde surge la motivaci�n del Proyecto Fin de Carrera (en adelante PFC), que consistir� en realizar un cuadro de mando para la visualizaci�n de la calidad del software, integrado con el entorno MEDUSAS.