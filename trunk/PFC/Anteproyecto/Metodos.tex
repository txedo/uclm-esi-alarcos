\section{M�todo y fases de trabajo}

Dada la naturaleza del proyecto, en el que se ir�n definiendo requisitos y funcionalidades a lo largo de todo su ciclo de vida, se ha optado por utilizar una metodolog�a de desarrollo de software gen�rica que permita adaptarse a este caso de estudio. Por estas razones, se ha seleccionado el \textbf{Proceso Unificado de Desarrollo} (en adelante PUD) como metodolog�a de trabajo.

El PUD \cite{jac00} es una evoluci�n del Proceso Unificado de Rational, que define un ``conjunto de actividades necesarias para transformar los requisitos de usuario en un sistema software''. Es un marco gen�rico que puede especializarse para una gran variedad de sistemas de software cualesquiera que sean el �rea de aplicaci�n o el tama�o del proyecto. Sus principales caracter�sticas son las siguientes:
\begin{myitemize}
	\item \textbf{Dirigido por casos de uso}. Para poder desarrollar un sistema es necesario saber qu� necesitan sus usuarios. Un usuario puede ser un ser humano u otro sistema que interacciona con el sistema que se est� desarrollando. Las necesidades de un usuario se denominan requisitos funcionales y se representan por medio de casos de uso.
	
	Los casos de uso gu�an el proceso de desarrollo desde la especificaci�n de requisitos hasta las pruebas y se utilizan para crear los modelos que permitan la construcci�n e implementaci�n de los mismos. Todos los casos de uso juntos constituyen el \textbf{modelo de casos de uso}.
	
	\item \textbf{Centrado en la arquitectura}. Un sistema software puede contemplarse desde varios puntos de vista. La arquitectura software incluye los aspectos est�ticos y din�micos m�s significativos del sistema y debe estar profundamente relacionada con los casos de uso ya que debe permitir el desarrollo de los mismos. Por esta raz�n, la arquitectura y los casos de uso deben desarrollarse en paralelo.
	
	\item \textbf{Iterativo e incremental}. Partiendo del dicho latino \textit{divide et impera} (en castellano, \textit{divide y vencer�s}), el ciclo de vida del proyecto (Fig. \ref{fig:mapa-conceptual-pud}) se divide en ciclos. Cada ciclo consta de cuatro fases que a su vez se dividen en iteraciones. Las fases de un ciclo son las siguientes:
	\begin{myitemize}
		\item \textbf{Inicio o Factibilidad}. Se obtiene un modelo de casos de uso simplificado, se identifican riesgos potenciales y se estima el proyecto de manera aproximada.
		
		\item \textbf{Elaboraci�n}. Se mejora el modelo de casos de uso y se dise�a la arquitectura del sistema. A continuaci�n, se desarrollan los casos de uso m�s cr�ticos que se identificaron en la fase de inicio y se obtiene una \textit{l�nea base} de la arquitectura para planificar las actividades y estimar los recursos necesarios para terminar el proyecto.
		
		\item \textbf{Construcci�n}. Se crea el producto en base a las dos fases anteriores.
		
		\item \textbf{Transici�n}. Implica la correcci�n de errores y \textit{bugs}, as� como el mantenimiento del sistema.
	\end{myitemize}
	
	 En cierto modo, una iteraci�n se corresponde con un mini-proyecto que resulta en un \textit{incremento}. En cada iteraci�n se aborda un conjunto de casos de uso que ampl�an la funcionalidad global del producto desarrollado y trata los riesgos m�s importantes. Si la iteraci�n cumple sus objetivos se contin�a con la pr�xima, en caso contrario se vuelve a las primeras etapas de la misma y se toman otras decisiones que permitan cumplir sus objetivos. Las iteraciones deben ser seleccionadas y ejecutarse de forma controlada y siguiendo un flujo de trabajo. Este flujo de trabajo consta de cinco etapas:
	\begin{myitemize}
		\item \textbf{Requisitos}. El equipo de desarrollo y los clientes establecen comunicaciones por diferentes medios para especificar qu� es lo que el usuario espera del sistema que se va a desarrollar.
		
		\item \textbf{An�lisis}. Se identifican y especifican los casos de uso relevantes.
		
		\item \textbf{Dise�o}. Se crea un dise�o utilizando la arquitectura seleccionada como gu�a para tratar de dotar al sistema de las funcionalidades representadas por los casos de uso identificados en la etapa de an�lisis.
		
		\item \textbf{Implementaci�n}. Se implementan las decisiones tomadas en la etapa de dise�o, con el fin de construir las funcionalidades que satisfacen los casos de uso identificados en la etapa de an�lisis.
		
		\item \textbf{Pruebas}. Se verifica que los cambios realizados en el sistema satisfacen los casos de uso.
	\end{myitemize}
\end{myitemize}


\imagenBorde{./images/mapa-conceptual-pud.png}{0.46}{Mapa conceptual del PUD}{fig:mapa-conceptual-pud}
