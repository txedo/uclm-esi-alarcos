% clase article a una cara, con letra de 12pt, en papel A4
% y con p�gina separada para el t�tulo
\documentclass[oneside,12pt,a4paper,titlepage]{article}

% Carga de paquetes necesarios. "customarticle" es un paquete personalizado
\usepackage[spanish]{babel} 
\RequirePackage[T1]{fontenc}
\RequirePackage[ansinew]{inputenx}
\usepackage{array}
\usepackage{graphicx}
\usepackage{pifont}
\usepackage[usenames,dvipsnames]{color}
\usepackage{amsmath}
\usepackage{hyperref}
\usepackage{colortbl}
\usepackage{makeidx}
\hypersetup{bookmarksopen,bookmarksopenlevel=3,linktocpage,colorlinks,urlcolor=blue,citecolor=blue,linkcolor=blue,filecolor=blue,pdfnewwindow,
	pdftitle={Cuadro de Mando para la Visualizaci�n de la Calidad del Software, Integrado con un Entorno de Medici�n}, 
	pdfsubject={Anteproyecto}
	pdfauthor={Jose Domingo L�pez L�pez}}
\usepackage[spanish,caption,counter,title,fancy]{customarticle}

\makeindex

% Macro para insertar una imagen
%       Uso: \imagen{nombreFichero}{Factor escala}{Caption (leyenda)}{Label (identificador para referenciarla)}
\def\imagen#1#2#3#4{
 \begin{figure}[h]
 \begin{center}
   \includegraphics[keepaspectratio,scale=#2]{#1}
 \caption {#3}
 \label{#4}
 \end{center}
 \end{figure}
}

% Definici�n de una lista personalizada 
\newenvironment{myitemize}%
{\begin{list}{\textbullet}%
{\settowidth{\labelwidth}{\textbullet}
\addtolength{\topsep}{-1.0ex}
\addtolength{\itemsep}{\dimexpr 0ex plus 0.15ex minus 0.25ex}
\setlength{\parsep}{0.0ex}
\setlength{\partopsep}{1.75ex}
\setlength{\parskip}{0.0ex}
}}%
{\end{list}}

\begin{document}

% En las p�ginas de portada e �ndices, no hay encabezado ni pie de p�gina
\pagestyle{empty}
\begin{titlepage}
	\begin{center}
		\includegraphics[scale=0.25, keepaspectratio]{./images/esi_bw.png} \\
		\vspace{20mm}
  	{\large \textbf{UNIVERSIDAD DE CASTILLA-LA MANCHA}} \\
  	\bigskip
  	{\large \textbf{ESCUELA SUPERIOR DE INFORM�TICA}} \\
  	\vspace{25mm}
  	{\large \textbf{INGENIER�A}}\\
  	\bigskip
  	{\large \textbf{EN INFORM�TICA}}\\
  	\vspace{25mm}
  	{\large \textbf{ANTEPROYECTO}}\\
  	\vspace{15mm}
  	{CUVICA\\\textbf{Cu}adro de Mando para la \textbf{Vi}sualizaci�n de la \textbf{Ca}lidad del Software,\\ Integrado con un Entorno de Medici�n}\\
  	\vspace{15mm}
  	{Jose Domingo L�pez L�pez}\\
  	\vspace{20mm}
  	\hfill
%	  \newbox{\editores}
%		\sbox{\editores}{Tutor acad�mico: M\textordfeminine{} �ngeles Moraga de la Rubia}
%  	\begin{minipage}[b][]{\wd\editores}
%	  	Director: Mois�s Rodriguez Monje\\
%	  	\usebox{\editores}
%		\end{minipage}
%		\vspace{\parskip}
  	\begin{flushright}
  		\textbf{Octubre, 2010}
  	\end{flushright}
	\end{center}
\end{titlepage}
% Derechos de autor
\parskip=10pt
\mbox{}
\vfill{}
\begin{small}
	\noindent Cuadro de Mando para la Visualizaci�n de la Calidad del Software, Integrado con un Entorno de Medici�n

	\noindent\copyright~ 2010 Jose Domingo L�pez L�pez \\
	\phantom{\copyright~ 2010} Mar�a de los �ngeles Moraga de la Rubia \\
	\phantom{\copyright~ 2010} Mois�s Rodriguez Monje

	\noindent En virtud del Art�culo 7 del Real Decreto Legislativo 1/1996, de 12 de abril por el que se aprueba el Texto Refundido de la Ley de Propiedad Intelectual, modificado por la Ley 23/2006 de 7 de julio, este PFC se considera una obra en colaboraci�n entre las diferentes partes. Por tanto la propiedad intelectual de este PFC (V�ase Anexo \ref{anexo:cpi}) de manera aislada, sin ser integrado con las herramientas y entornos de la empresa Alarcos Quality Center, ser� compartida con iguales porcentajes entre el alumno, el director y el tutor acad�mico.

	\noindent Este documento ha sido compuesto con \LaTeX{}. Im�genes generadas con GIMP 2.6.
\end{small}
\clearpage
% En los �ndices numeramos las p�ginas en n�meros romanos
\pagestyle{plain}
\pagenumbering{Roman}
% �ndice de secciones
\tableofcontents
\clearpage

\parskip=10pt
\pagenumbering{arabic}
\pagestyle{fancy}
% Aqui se incluyen los archivos .tex que forman el documento
\section{Introducci�n}

A�os atr�s, las organizaciones y empresas se preocupaban de la calidad de los procesos que se segu�an para desarrollar software bas�ndose en la premisa de que \textit{``con un proceso de calidad se obtiene un producto de calidad''} \cite{fern�ndez-evaluacion}, pero dados los problemas que han surgido en sistemas ya implantados y en producci�n, y las p�rdidas que �stos han ocasionado, ha sido necesario actualizar la percepci�n que se tiene acerca de la calidad del software. Por ello, las organizaciones comenzaron a preocuparse por la calidad del producto software desde el punto de vista de los clientes y usuarios finales, abarcando un mayor n�mero de \textit{stakeholders}, con el fin de aumentar el grado de satisfacci�n y confianza por parte de los mismos.

Es por esto por lo que se tratado de asegurar la funcionalidad y usabilidad de los sistemas profundizando en �rea de \textit{testing}, pero se han quedado sin cubrir aspectos de seguridad y mantenibilidad. Y es que un producto software no es �nicamente el c�digo que se ejecuta sobre una m�quina, si no que lo son tambi�n los artefactos que se han ido produciendo a lo largo de todo el ciclo de vida: modelos, diagramas, documentos, etc.

Para paliar algunos de los problemas expuestos anteriormente, se comenzaron a utilizar dos est�ndares:
\begin{myitemize}
	\item \textbf{ISO/IEC 9126} \cite{iso-9126}, que define el modelo de calidad de un producto, incluyendo las caracter�sticas, subcaracter�sticas y m�tricas (de calidad interna, calidad externa y calidad en uso) que han de tenerse en cuenta, pero sin proporcionar una metodolog�a para su evaluaci�n. Las caracter�sticas que propone son funcionalidad, fiabilidad, usabilidad, eficiencia, mantenibilidad y portabilidad.
	\item \textbf{ISO/IEC 14598} \cite{iso-14598}, que proporciona gu�as y requisitos para el proceso de evaluaci�n, cubriendo as� los aspectos de los que carece la norma ISO/IEC 9126.
\end{myitemize}

Pero entre ambos est�ndares existen distintas incompatibilidades e inconsistencias -tales como diferencias en el vocabulario y t�rminos, la necesidad de a�adir recomendaciones, metodolog�as, gu�as y nuevas necesidades para especificar las dimensiones de la calidad del software, entre otros- que motivan la creaci�n de la familia \textbf{ISO/IEC 25000}, denominada SQuaRe (\underline{S}oftware product \underline{Qua}lity \underline{R}equirements and \underline{E}valuation).

%La norma ISO/IEC 25000, que consiste en 14 documentos agrupados en 5 vol�menes \cite{suryn2003iso}, establece criterios para la especificaci�n de requisitos de calidad de productos software y su evaluaci�n, un modelo de calidad, as� como m�tricas recomendadas para los distintos atributos de dicho modelo, teniendo en cuenta a todos los \textit{stakeholders} (clientes, usuarios finales y factor�as y empresas de desarrollo).
% TODO En este p�rrafo sacar lo del post de mois�s
% http://www.iso15504.es/index.php/component/content/article/82-blog/165-la-familia-de-normas-iso-25000-y-su-relacion-con-iso-15504-spice.html

La aparici�n de esta norma, cuyo objetivo es reemplazar a las normas ISO/IEC 9126 e ISO/IEC 14598, junto con las nuevas necesidades que surgen en las empresas para controlar y evaluar la calidad de los desarrollos inform�ticos, motivan la aparici�n del proyecto \textbf{MEDUSAS} (\underline{M}ejora y \underline{E}valuaci�n del \underline{D}ise�o, \underline{U}sabilidad, \underline{S}eguridad y \underline{M}antenibilidad del \underline{S}oftware). MEDUSAS (Fig. \ref{fig:arch-medusas}) es un entorno metodol�gico e instrumental basado en la ISO/IEC 25000 para satisfacer dichas necesidades, y evaluar la calidad del software teniendo en cuenta todos y cada uno de sus artefactos: c�digo fuente, dise�o, diagramas, modelos, etc.

\imagen{./images/comp_medusas.png}{0.4}{Arquitectura del proyecto MEDUSAS}{fig:arch-medusas}

Aqu� es donde surge la motivaci�n del Proyecto Fin de Carrera (en adelante PFC), que consistir� en realizar un cuadro de mando para la visualizaci�n de la calidad del software, integrado con el entorno MEDUSAS.

% TODO Meter aqu� un p�rrafo de la importancia de la representaci�n a distintos niveles de calidad (estrat�gico, operativo y t�ctico), que se enganche mejor con el PFC (visualizaci�n, obtenci�n, automatizaci�n...)
\clearpage
\section{Objetivos}

\subsection{Objetivo principal}

\subsection{Objetivos secundarios}
\clearpage
\section{M�todo y fases de trabajo}

Dada la naturaleza del proyecto, en el que se ir�n definiendo requisitos y funcionalidades a lo largo de todo su ciclo de vida, se ha optado por utilizar una metodolog�a de desarrollo de software gen�rica que permita adaptarse a este caso de estudio. Por estas razones, se ha seleccionado el \textbf{Proceso Unificado de Desarrollo} (en adelante PUD) como metodolog�a de trabajo.

El PUD \cite{jac00} es una evoluci�n del Proceso Unificado de Rational, que define un ``conjunto de actividades necesarias para transformar los requisitos de usuario en un sistema software''. Es un marco gen�rico que puede especializarse para una gran variedad de sistemas de software cualesquiera que sean el �rea de aplicaci�n o el tama�o del proyecto. Sus principales caracter�sticas son las siguientes:
\begin{myitemize}
	\item \textbf{Dirigido por casos de uso}. Para poder desarrollar un sistema es necesario saber qu� necesitan sus usuarios. Un usuario puede ser un ser humano u otro sistema que interacciona con el sistema que se est� desarrollando. Las necesidades de un usuario se denominan requisitos funcionales y se representan por medio de casos de uso.
	
	Los casos de uso gu�an el proceso de desarrollo desde la especificaci�n de requisitos hasta las pruebas y se utilizan para crear los modelos que permitan la construcci�n e implementaci�n de los mismos. Todos los casos de uso juntos constituyen el \textbf{modelo de casos de uso}.
	
	\item \textbf{Centrado en la arquitectura}. Un sistema software puede contemplarse desde varios puntos de vista. La arquitectura software incluye los aspectos est�ticos y din�micos m�s significativos del sistema y debe estar profundamente relacionada con los casos de uso ya que debe permitir el desarrollo de los mismos. Por esta raz�n, la arquitectura y los casos de uso deben desarrollarse en paralelo.
	
	\item \textbf{Iterativo e incremental}. Partiendo del dicho latino \textit{divide et impera} (en castellano, \textit{divide y vencer�s}), el ciclo de vida del proyecto (Fig. \ref{fig:mapa-conceptual-pud}) se divide en ciclos. Cada ciclo consta de cuatro fases que a su vez se dividen en iteraciones. Las fases de un ciclo son las siguientes:
	\begin{myitemize}
		\item \textbf{Inicio o Factibilidad}. Se obtiene un modelo de casos de uso simplificado, se identifican riesgos potenciales y se estima el proyecto de manera aproximada.
		
		\item \textbf{Elaboraci�n}. Se mejora el modelo de casos de uso y se dise�a la arquitectura del sistema. A continuaci�n, se desarrollan los casos de uso m�s cr�ticos que se identificaron en la fase de inicio y se obtiene una \textit{l�nea base} de la arquitectura para planificar las actividades y estimar los recursos necesarios para terminar el proyecto.
		
		\item \textbf{Construcci�n}. Se crea el producto en base a las dos fases anteriores.
		
		\item \textbf{Transici�n}. Implica la correcci�n de errores y \textit{bugs}, as� como el mantenimiento del sistema.
	\end{myitemize}
	
	 En cierto modo, una iteraci�n se corresponde con un mini-proyecto que resulta en un \textit{incremento}. En cada iteraci�n se aborda un conjunto de casos de uso que ampl�an la funcionalidad global del producto desarrollado y trata los riesgos m�s importantes. Si la iteraci�n cumple sus objetivos se contin�a con la pr�xima, en caso contrario se vuelve a las primeras etapas de la misma y se toman otras decisiones que permitan cumplir sus objetivos. Las iteraciones deben ser seleccionadas y ejecutarse de forma controlada y siguiendo un flujo de trabajo. Este flujo de trabajo consta de cinco etapas:
	\begin{myitemize}
		\item \textbf{Requisitos}. El equipo de desarrollo y los clientes establecen comunicaciones por diferentes medios para especificar qu� es lo que el usuario espera del sistema que se va a desarrollar.
		
		\item \textbf{An�lisis}. Se identifican y especifican los casos de uso relevantes.
		
		\item \textbf{Dise�o}. Se crea un dise�o utilizando la arquitectura seleccionada como gu�a para tratar de dotar al sistema de las funcionalidades representadas por los casos de uso identificados en la etapa de an�lisis.
		
		\item \textbf{Implementaci�n}. Se implementan las decisiones tomadas en la etapa de dise�o, con el fin de construir las funcionalidades que satisfacen los casos de uso identificados en la etapa de an�lisis.
		
		\item \textbf{Pruebas}. Se verifica que los cambios realizados en el sistema satisfacen los casos de uso.
	\end{myitemize}
\end{myitemize}


\imagenBorde{./images/mapa-conceptual-pud.png}{0.46}{Mapa conceptual del PUD}{fig:mapa-conceptual-pud}

\clearpage
\section{Medios que se pretenden utilizar}


\clearpage

\clearpage\phantomsection
\pagestyle{plain}
% A�ade la bibliograf�a al �ndice
\addcontentsline{toc}{section}{\bibname}
% Bibliograf�a
\bibliographystyle{plain}
\bibliography{Bibliografia}
\clearpage
% A�ade los ap�ndices al �ndice
\addcontentsline{toc}{section}{Anexos}
% Ap�ndices
\section*{Anexos}

\appendix

\section{Solicitud de evaluaci�n}

\newsavebox{\caja}
\sbox{\caja}{\textbf{Escuela Superior de Inform�tica}}
\newlength{\anchoCaja}
\setlength{\anchoCaja}{\wd\caja}
\noindent
\begin{minipage}{\anchoCaja}
	\begin{center}
	\noindent\includegraphics[keepaspectratio,scale=0.25]{./images/esi_bw.png}\\
		\noindent\textbf{Escuela Superior de Inform�tica}
	\end{center}
\end{minipage}

\vspace{10mm}
\noindent D. Jose Domingo L�pez L�pez, con D.N.I. n\textordmasculine{} 71219116-F, alumno de la Escuela Superior de Inform�tica.

\vspace{5mm}
\noindent EXPONE:\\
\textemdash{} Que se encuentra matriculado del PFC.

\vspace{5mm}
\noindent SOLICITA:\\
\textemdash{} Que la Comisi�n Acad�mica eval�e el anteproyecto de fin de carrera que acompa�a, titulado \textit{DESGLOSA: Un sistema de visualizaci�n 3D para dar soporte al Desarrollo Global de Software\\}.

\vspace{10mm}
\noindent Ciudad Real, a \today{}\\

\newlength{\factor}
\setlength{\factor}{0.32\textwidth}
\noindent 
\begin{minipage}{\factor}
El alumno,\\
\\
\\
\\
Fdo.:
\end{minipage}
\hfill
\begin{minipage}{\factor}
V\textordmasculine B\textordmasculine{} El director,\\
\\
\\
\\
Fdo.:
\end{minipage}
\hfill
\begin{minipage}{\factor}
V\textordmasculine B\textordmasculine{} El tutor acad�mico\\
\\
\\
\\
Fdo.:
\end{minipage}

\vspace{10mm}
\begin{center}
	\noindent SR. DIRECTOR DE LA ESCUELA SUPERIOR DE INFORM�TICA
\end{center}

\clearpage

\section{Contrato de Propiedad Intelectual}\label{anexo:cpi}

\input{Anexo_CPI.tex}


\end{document}
