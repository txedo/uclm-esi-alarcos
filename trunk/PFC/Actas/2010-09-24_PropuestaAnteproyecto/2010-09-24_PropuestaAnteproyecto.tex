\documentclass[a4paper,11pt]{article}

\usepackage[spanish]{babel}
\usepackage[T1]{fontenc}
\usepackage{lmodern}
\usepackage{hyperref}
\usepackage{graphicx}

\setlength{\headheight}{0 pt}
\setlength{\headsep}{0 pt}
\setlength{\textheight}{\dimexpr \paperheight-3cm-\headheight-\headsep-\footskip-1.5cm}
\setlength{\topmargin}{\dimexpr 3cm-1in}
\setlength{\textwidth}{\dimexpr \paperwidth-2.75cm*2}
\setlength{\oddsidemargin}{\dimexpr 2.75cm-1in}

% Macro para insertar una imagen
%       Uso: \imagen{nombreFichero}{Factor escala}{Caption (leyenda)}{Label (identificador para referenciarla)}
%-------------------------------
\def\imagen#1#2#3#4{
 \begin{figure}[h]
 \begin{center}
   \scalebox{#2}{\includegraphics{#1}}
 \caption {#3}
 \label{#4}
 \end{center}
 \end{figure}
}

\begin{document}
\title{{\scshape \large 3\textordfeminine{} Reuni�n de PFC \\
\huge Primera Versi�n del Anteproyecto}}
\author{L�pez J.D.; Rodriguez M.}
\date{\today}
\maketitle

\begin{abstract}
	Los temas planeados para su tratamiento en esta convocatoria consisten en revisar una primera versi�n del anteproyecto, proponer un esqueleto para la memoria del PFC y justificar las librer�as y tecnolog�as seleccionadas.
\end{abstract}

\tableofcontents
%\clearpage

%\section{Introducci�n}

El sistema a desarrollar se puede categorizar como una \textbf{Aplicaci�n de Internet Enriquecida}, en ingl�s \textbf{R}ich \textbf{I}nternet \textbf{A}pplication o \textbf{RIA}. Una RIA, concepto cada vez m�s com�n en la era de la Web 2.0, \cite{wikipedia:ria} es una aplicaci�n web que, siendo accedida a trav�s de un navegador web est�ndar, trata de satisfacer las necesidades del usuario como si de una aplicaci�n de escritorio se tratase, combinando as� las ventajas de ambos paradigmas (aplicaci�n web y aplicaci�n de escritorio) para maximizar la experiencia del mismo.

Una RIA se difiere de una p�gina web tradicional en el sentido de que no hay una recarga continua de p�ginas cada vez que el usuario pulsa sobre un enlace. Esto evita producir un alto tr�fico de informaci�n entre el cliente y el servidor por grande o peque�a que sea la informaci�n que se desea actualizar. Adem�s, una RIA posee una gran capacidad multimedia ya que es posible empotrar distintos tipos de objetos (applets, v�deos, sonidos) en ella sin necesidad de utilizar programas externos para su visualizaci�n.

Una vez establecidas las caracter�sticas que debe cumplir una RIA y estudiadas las capacidades de visualizaci�n que posee Sonar \cite{homepage:sonar}, se proponen un conjunto de tecnolog�as, librer�as y servicios que permitir�n implementar un sistema que satisfaga los requisitos de usuario correspondientes.
%\section{Tecnolog�as}

La capa de conocimiento del sistema estar� desarrollado en Java, por lo que se propone JSP (Java Server Pages) como tecnolog�a principal para el desarrollo de la aplicaci�n Web. Se har� uso de otras tecnolog�as comunes entre la Web 2.0 como CSS, JavaScript y Ajax. No obstante, con un prop�sito investigador, se propone tambi�n el uso de HTML 5, Struts 2 e Hibernate.



\subsection{HTML 5}

HTML 5 (HyperText Markup Language, versi�n 5) \cite{wikipedia:html5} es la quinta revisi�n de HTML.  Esta nueva versi�n incorpora de forma nativa, es decir, sin depender de programas externos o \textit{plugins} del navegador, reproducci�n de audio y v�deo, edici�n de documentros, \textit{drag\&drop}, un elemento \textit{canvas} sobre el que se pueden pintar gr�ficos en 2 y 3 dimensiones, etc. Esta �ltima caracter�stica hace que sea posible el uso de WebGL (ver secci�n \ref{sec:webgl}).



\subsection{Struts 2}

Struts 2 es la nueva versi�n del \textit{framework} de desarrollo Web Java Apache Struts. Struts 2 est� basado en el patr�n MVC (Modelo-Vista-Controlador), una arquitectura que busca reducir el acoplamiento dividiendo las responsabilidades en 3 capas claramente diferenciadas:

\begin{itemize}
	\item El modelo, que hace referencia a los datos que maneja la aplicaci�n y las reglas de negocio que operan sobre ellos.
	\item La vista, encargada de generar la interfaz con la que la aplicaci�n interacciona con el usuario.
	\item El controlador, que comunica la vista y el modelo respondiendo a eventos generados por el usuario en la vista, invocando cambios en el modelo, y devolviendo a la vista la informaci�n del modelo necesaria para que pueda generar la respuesta adecuada para el usuario.
\end{itemize}
    
No obstante, tambi�n es posible dise�ar una aplicaci�n Web mediante el patr�n MVC utilizando JSP y servlets. En este caso, las p�ginas JSP tomar�an el rol de la vista, los servlets el del controlador y los POJO (Plan Old Java Object) el del modelo. El modelo, a su vez, suele estar dividido en dos subcapas siguiendo el patr�n DAO (Data Access Object).

En la aplicaci�n Web, Struts 2 se emplear� en el desarrollo de determinados casos de uso extra�dos de la especificaci�n de requisitos de usuario, identificando dichos casos de uso como \textit{acciones} de Struts.



\subsection{Hibernate}

Hibernate es un \textit{framework} de persistencia que permite establecer un \textit{mapeo objeto-relacional} para la plataforma Java y sistemas basados en SQL, es decir, permite relacionar una clase Java con una tabla de una base de datos.

Esta herramienta se utilizar� para gestionar toda la persistencia del sistema Medusas: informaci�n extra�da de distintos proyectos, gesti�n de usuarios, gesti�n de la aplicaci�n Web, etc.
%\section{Librer�as}

\subsection{Desarrollo Web}

\subsubsection{DWR: Direct Web Remoting}

\noindent \textbf{Versi�n:} 2.0 stable \\
\noindent \textbf{Licencia:} Apache Software License v2 \cite{license:apache-v2} \\
\noindent \textbf{P�gina Web:} \cite{homepage:dwr}

Easy Ajax for Java



\subsubsection{JQuery}

\noindent \textbf{Versi�n:} \\
\noindent \textbf{Licencia:} \\
\noindent \textbf{P�gina Web:} 

Descripci�n
% http://jquery.com/
% http://es.wikipedia.org/wiki/JQuery



\subsubsection{JQuery UI}

\noindent \textbf{Versi�n:} \\
\noindent \textbf{Licencia:} \\
\noindent \textbf{P�gina Web:} 

Descripci�n
% http://jqueryui.com/
% http://es.wikipedia.org/wiki/JQuery_UI


\subsubsection{YUI: Yahoo! UI Library}

\noindent \textbf{Versi�n:} \\
\noindent \textbf{Licencia:} BSD License \\
\noindent \textbf{P�gina Web:} 

Descripci�n
% http://developer.yahoo.com/yui/
% http://es.wikipedia.org/wiki/Yahoo_User_Interface



\subsubsection{GWT: Google Web Toolkit}

\noindent \textbf{Versi�n: 2.0} \\
\noindent \textbf{Licencia:} Apache Software License v2 \cite{license:apache-v2} \\
\noindent \textbf{P�gina Web:} 

Descripci�n
% http://code.google.com/intl/es-ES/webtoolkit/
% http://es.wikipedia.org/wiki/Google_Web_Toolkit





\subsection{Visualizaci�n}

\subsubsection{Google Visualization API}

\noindent \textbf{Versi�n} \\
\noindent \textbf{Licencia:} \\
\noindent \textbf{P�gina Web:} 

Descripci�n



\subsubsection{Protovis}

\noindent \textbf{Versi�n} \\
\noindent \textbf{Licencia:} \\
\noindent \textbf{P�gina Web:} 

Descripci�n



\subsubsection{The javaScript InfoVis Toolkit}

\noindent \textbf{Versi�n} \\
\noindent \textbf{Licencia:} \\
\noindent \textbf{P�gina Web:} 

Descripci�n



\subsubsection{JFreeChart}

\noindent \textbf{Versi�n} \\
\noindent \textbf{Licencia:} \\
\noindent \textbf{P�gina Web:} 

Descripci�n



\subsubsection{PlotKit}

\noindent \textbf{Versi�n} \\
\noindent \textbf{Licencia:} \\
\noindent \textbf{P�gina Web:} 

Descripci�n



\subsubsection{JOGL: JavaOpenGL}

\noindent \textbf{Versi�n} \\
\noindent \textbf{Licencia:} \\
\noindent \textbf{P�gina Web:} 

Descripci�n



\subsubsection{WebGL: OpenGL ES 2.0 for the Web} \label{sec:webgl}

\noindent \textbf{Versi�n} \\
\noindent \textbf{Licencia:} \\
\noindent \textbf{P�gina Web:} 

Descripci�n
% http://www.khronos.org/webgl/
% http://en.wikipedia.org/wiki/WebGL
% http://khronos.org/webgl/wiki/Main_Page
% http://www.khronos.org/webgl/wiki/Getting_a_WebGL_Implementation

%\newpage
%\bibliographystyle{plain}
%\pagestyle{plain}
%\bibliography{bibliografia}

\section{Revisi�n del Anteproyecto}

En esta reuni�n se revisar� la primera versi�n del anteproyecto, que previamente ha sido enviada por correo electr�nico al director del PFC.

La estructura que se ha seguido para la elaboraci�n del anteproyecto ha sido seg�n la especificaci�n propuesta en la normativa aprobada por la Junta de Centro el 8 de Noviembre de 2007, vigente hasta el d�a de hoy. En dicha normativa se indica que el anteproyecto debe ocupar un m�ximo de 10 p�ginas. suponiendo que esa limitaci�n se refiere al contenido del anteproyecto, excluyendo portada, �ndice, etc., el documento elaborado consta de 17 p�ginas, de las cuales:
\begin{itemize}
	\item Una p�gina corresponde a la portada del anteproyecto, que sigue el formato propuesto para la portada de la documentaci�n final.
	\item Una p�gina corresponde a la contraportada, en la que se indica el nombre del alumno, del tutor acad�mico y del director del PFC.
	\item Una p�gina corresponde al �ndice de contenidos.
	\item Diez p�ginas corresponden a los contenidos del anteproyecto, subdivididos a su vez en los puntos que se especifican en la misma normativa: objetivos, m�todo y fases de trabajo y medios que se pretenden utilizar. Adicionalmente, se ha a�adido una peque�a secci�n de introducci�n.
	\item Una p�gina corresponde a la bibliograf�a y referencias consultadas en la elaboraci�n del mismo.
	\item Tres p�ginas corresponden al contrato de propiedad intelectual, que deber� ser firmado por el alumno, el tutor acad�mico y el director del PFC.
\end{itemize}


\section{Esqueleto de la memoria del PFC}

Actualmente se est� trabajando en el dise�o de la plantilla \LaTeX{}, por lo que no se tiene fijado a�n un modelo de documentaci�n.


\section{Justificaci�n de librer�as y tecnolog�as seleccionadas}

Este punto se trata en 
la secci�n \textit{Medios que se pretenden utilizar} del anteproyecto.

\end{document}