\chapter{�Qu� es \LaTeX{}?}
\label{ch:Introduccion}

\LaTeX{} (escrito LaTeX en texto plano) es un sistema de composici�n de textos, orientado especialmente a la creaci�n de libros, documentos cient�ficos y t�cnicos que contengan f�rmulas matem�ticas.\footnote{\url{http://es.wikipedia.org/wiki/LaTeX}}

\LaTeX{} est� formado por un gran conjunto de macros de \TeX{},\footnote{\TeX{} es un lenguaje de programaci�n creado por D.~Knuth para la composici�n tipogr�fica de documentos t�cnicos con un acabado de gran calidad.} escrito por Leslie Lamport\index{Lamport, L.} en 1984, con la intenci�n de facilitar el uso del lenguaje de composici�n tipogr�fica, creado por Donald Knuth.\index{Knuth, D.} Es muy utilizado para la composici�n de art�culos acad�micos, tesis y libros t�cnicos, dado que la calidad tipogr�fica de los documentos realizados con \LaTeX{} es comparable a la de una editorial cient�fica de primera l�nea.

En \TeX{} y \LaTeX{} las palabras reservadas o <<comandos>> del lenguaje est�n precedidos por la barra inclinada o \emph{backslash}\index{backslash@\verb+\textbackslash+} (\textbackslash). Otros caracteres especiales\footnote{Con sisgnificado especial para el processador \LaTeX{}.} son: \# \$ \% \textasciicircum \& \_ \{ \} \~{}. Para escribir estos caracteres se emplea: 
\begin{verbatim} 
	\# \$ \% \textasciicircum \& \_ \{ \} \~
\end{verbatim}
% Notar que todo lo que se escribe entre las intrucciones \begin{verbatim} y \end{verbatim} LaTeX lo escribe tal cual y con un tipo de letra de m�quina de escribir. Lo mismo ocurre con el texto escrito con la instrucci�n \veb+texto+.

Una o m�s l�neas en blanco se tratan como un salto de p�rrafo, aunque se puede provocar un salto de l�nea mediante doble \emph{backslash} \verb+\\+. Aunque en este caso no se considera un p�rrafo nuevo.

Como sucede con la mayor�a de lenguajes de programaci�n las palabras reservadas pertenecen al ingl�s (entre otras cosas porque es la lengua materna de sus creadores). Tambi�n todos los mensajes de error y avisos del procesador \LaTeX{} dirigidos al usuario son textos en ingl�s que deben ser comprendidos para subsanar los posibles errores. Esto puede suponer una dificultad adicional para alguien que no tenga conocimientos de dicho idioma, pero no deber�a serlo ya que el ingl�s es la \emph{lingua franca} de la ciencia y la tecnolog�a, �reas a las que est� especialmente dirigido \TeX{}.

La versi�n utilizada para generar este documento es \LaTeXe. \LaTeX{} es software libre bajo licencia LPPL.