% Se debe recordar que aunque en el tipo de documento report se puede incluir un abstract (resumen). En la clase book este entorno no est� definido.
%====================================
%====================================
%.... RESUMEN (opcional, 1 p�g.)
\selectlanguage{spanish}%
\begin{abstract}
(... versi�n del resumen en espa�ol ...)
Este es un documento en el que se refunden, a modo de plantilla de PFC, todos los ejemplos empleados durante el curso de \LaTeX{}. El usuario puede emplearlo para modificarlo a su gusto en la elaboraci�n y personalizaci�n de su proyecto.
\end{abstract}

%====================================
%====================================
%.... ABSTRACT (opcional, 1 p�g.)
% Aqu� se muestra un ejemplo completo en el que se a�ade una versi�n en ingl�s del resumen.
\selectlanguage{english}% Selecci�n de idioma ingl�s.
\begin{abstract}
... english version for the abstract ...
\end{abstract}

\selectlanguage{spanish}% Para el resto del documento del idioma es espa�ol.
%====================================
%====================================
%.... AGRADECIMIENTOS (opcional, 1 p�g. o m�s, recomendable hacerloe en una s�lo)
% Los agradecimientos son m�s extensos que la dedicatoria como se muestra en el ejemplo.
\chapter*{Agradecimientos} % Opci�n con * para que no aparezca en TOC
En este trabajo quiero expresar mi agradecimiento a la Escuela Superior de Inform�tica (ESI) de la Universidad de Castilla-La Mancha por darme la oportunidad de impartir este curso de \LaTeX{} y poner los medios materiales e infraestructuras para llevarlo a cabo. Tambi�n quiero agradecer a los alumnos del curso sus aportaciones para mejorar los contenidos y el inter�s demostrado haciendo que la tarea de impartir sea muy gratificante. Por �ltimo no quiero olvidar a los miembros del PAS de la ESI que han realizado una gran labor para solucionar todos los aspectos administrativos del curso. A todos ellos mi sincero agradecimiento.
