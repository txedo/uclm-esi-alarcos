\documentclass[10pt,xcolor=pdftex]{beamer}

\usepackage[ansinew]{inputenx}			% C�digo de caracteres de teclado
\usepackage[spanish,english]{babel}		% Internacionalizaci�n de LaTeX
\usepackage[T1]{fontenc}      			% Codificaci�n de fundiciones
\usepackage{microtype}        			% Mejoras para pdflatex

\usepackage{graphicx}
\graphicspath{{images/}}

\usetheme{Warsaw}

\title[DESGLOSA: Sistema de Visualizaci�n 3D para DGS]{DESGLOSA\\Un sistema de visualizaci�n 3D\\para dar soporte al Desarrollo Global de Software}
\author{Jose Domingo L�pez L�pez}
\institute{Escuela Superior de Inform�tica\\Universidad de Castilla-La Mancha}
\date{3 de Febrero de 2012}

\begin{document}

\AtBeginSection[]{
  \begin{frame}
    \frametitle{Contenidos}
    \tableofcontents[currentsection,hideothersubsections]
  \end{frame}
}

\AtBeginSubsection[]
{
  \begin{frame}
    \frametitle{Contenidos}
    \tableofcontents[currentsection,currentsubsection,hideothersubsections]
  \end{frame}
}

\begin{frame}
\titlepage
\end{frame}

\begin{frame}
  \frametitle{Contenidos}
  \tableofcontents[hideallsubsections]
\end{frame} 


\section{Introducci�n}
\begin{frame}{Introducci�n}
\uncover<2->
{appear from slide 2 on\\}
\uncover<3-4>
{appears from 3 to slide 4\\}
\uncover<4>{appears on slide 4\\}
\uncover<3->{appears from slide 3 on\\}
\end{frame}

\begin{frame}{Problema a resolver}
This is a short introduction to Beamer class.
\end{frame}

\begin{frame}{Qu� se propone}
This is a short introduction to Beamer class.
\end{frame}




\section{Motivaci�n y objetivos}
\subsection{Motivaci�n}
\begin{frame}{Motivaci�n}
This is a short introduction to Beamer class.
\end{frame}

\subsection{Objetivos}
\begin{frame}{Objectivo principal}
This is a short introduction to Beamer class.
\end{frame}

\begin{frame}{Objetivos parciales}
This is a short introduction to Beamer class.
\end{frame}


\section{Estado del Arte}
\begin{frame}{Conocimientos te�ricos}
  \begin{block}{Pilares fundamentales en los que se basa el PFC}
    \begin{itemize}
    \pause \item Desarrollo Global de Software.
    \pause \item Medidas y calidad del software.
    \pause \item Visualizaci�n.
    \end{itemize}
  \end{block}
\end{frame}

\subsection{Desarrollo Global de Software (DGS)}
\begin{frame}{Desarrollo Global de Software}
This is a short introduction to Beamer class.
\end{frame}

\subsection{Medidas y calidad del software}
\begin{frame}{Medidas y calidad del software}
This is a short introduction to Beamer class.
\end{frame}

\subsection{Visualizaci�n}
\begin{frame}{Visualizaci�n}
This is a short introduction to Beamer class.
\end{frame}


\section{Metodolog�a}
\subsection{Marco de investigaci�n preliminar}
\subsection{Proceso Unificado de Desarrollo (PUD)}
\subsection{Marco tecnol�gico}


\section{Resultados}
\begin{frame}{Resultados}
Decir que se hizo en un ciclo pero que se pudo hacer en varios.
\end{frame}

\subsection{Trabajo inicial}
\begin{frame}{Trabajo inicial}
captura der equisitos
identificaci�n de requisitos funcionales y no funcionales
modelo de casos de uso preliminar, se determinan dos componentes
Glosario de terminos
Gesti�n del riesgo
Plan de iteraciones
\end{frame}

\subsection{Motor gr�fico}

\subsection{Aplicaci�n web}


\section{Demostraci�n de la herramienta}

\section{Conclusiones y trabajo futuro}
\subsection{Aspectos destacables de la soluci�n propuesta}

\subsection{Trabajo actual y futuras mejoras}

\subsection{Conocimientos adquiridos}


\end{document}