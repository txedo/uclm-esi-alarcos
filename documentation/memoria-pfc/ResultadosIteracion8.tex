En esta iteraci�n (Tabla \ref{table:ficha-iteracion-8}) se aborda:
\begin{itemize}
	\item An�lisis y dise�o de las funcionalidades relativas a la gesti�n de compa��as, factor�as, proyectos y subproyectos.
	\item Implementaci�n de la gesti�n y selecci�n de perfiles de visualizaci�n.
	\item Pruebas unitarias y funcionales relativas a la gesti�n y selecci�n de perfiles de visualizaci�n.
\end{itemize}


\subsubsection{Disciplina de an�lisis}
\label{sec:iteracion-8-analisis}

\imagenBorde{Cap5//it8-management-usecase-diagram}{0.9}{Diagrama de casos de uso para la gesti�n de compa��as y factor�as (CdU2.3 y CdU2.4)}{fig:it8-management-usecase-diagram}

\imagenBorde{Cap5//it8-company-create-communication-diagram}{0.9}{Diagrama de comunicaci�n para la creaci�n de una compa��a (CdU2.3)}{fig:it8-company-create-communication-diagram}


\subsubsection{Disciplina de dise�o}
\label{sec:iteracion-8-dise�o}

\imagenBorde{Cap5//it8-company-create-sequence-diagram}{0.9}{Diagrama de secuencia para la creaci�n de una compa��a (CdU2.3)}{fig:it8-company-create-sequence-diagram}

\imagenBorde{Cap5//it8-management-class-diagram}{0.9}{Diagrama de clases de dominio}{fig:it8-management-class-diagram}


\subsubsection{Disciplina de implementaci�n}
\label{sec:iteracion-8-implementacion}

implementaci�n de 2.10 mediante reflexion -- conlleva la generacion de la cadena json mediante la clase city




\subsubsection{Disciplina de pruebas}
\label{sec:iteracion-8-pruebas}

Durante esta disciplina se construyen los casos de prueba que sirven para
comprobar la funcionalidad de los requisitos implementados la disciplina de implementaci�n de esta misma iteraci�n.

Cabe destacar que uno de los casos de prueba consiste comprobar que los tipos de datos asociados los atributos de las entidades y las dimensiones de la met�fora sean compatibles.

No se incorporan m�s detalles al documento para no alargar la extensi�n del mismo.
