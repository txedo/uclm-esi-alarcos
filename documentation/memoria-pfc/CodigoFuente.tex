\chapter{Codigo Fuente}
\label{ch:codigofuente}


\section{Autenticaci�n y control de acceso con Spring Security}

Spring Security es el est�ndar \textit{de-facto} para la seguridad de aplicaciones basadas en Spring. Consiste en un \textit{framework} potente y configurable que proporciona mecanismos de autenticaci�n y permite implementar pol�ticas de control de acceso a directorios y recursos.

El sistema de autenticaci�n y control de acceso de este PFC se ha configurado e implementado de un modo muy gen�rico y satisfaciendo una pol�tica basa en grupos. Por tanto, dada la gran posibilidad de reutilizaci�n en mutitud de proyectos, se ha optado por incluir en este anexo los ficheros y clases m�s relevantes.

\subsection{Clases Java dise�adas para un GBAC (Group-Based Access Control}

El control de acceso basado en grupos (GBAC) extiende al modelo basado en roles (RBAC, en su acr�nimo ingl�s Role-Base Access Control) y ofrece m�s flexibilidad para modelar pol�ticas de acceso m�s complejas y exigentes. En este. En el caso del GBAC, los privilegios se otorgan al usuario en funci�n de los grupos a los que pertenece. Un usuario puede pertenecer a uno o varios
grupos, y un grupo puede tener asignados uno o varios roles. De aqu� se deduce que son necesarias tres clases para representar los conceptos de usuario, grupo y rol.


\subsubsection{Clase User.java}
\label{anexo:user}

En la implementaci�n de esta clase destacan sus anotaciones mediante JPA (Java Persistence API) y que implementa el \textit{interface} \textbf{UserDetails} provisto por Spring Security.

N�tese la relaci�n \textit{muchos-a-muchos} entre las clases \textit{User} y \textit{Group}.

\lstinputlisting[language=Java,caption={C�digo fuente de la clase User.java},label=code:User.java]{..//..//desglosa-web//src//main//java//es//uclm//inf_cr//alarcos//desglosa_web//model//User.java}


\subsubsection{Clase Group.java}
\label{anexo:group}

En la implementaci�n de esta clase destacan nuevamente sus anotaciones mediante JPA y el m�todo \textit{getAuthorities} que devuelve la lista de roles asignados al grupo.

N�tese la relaci�n \textit{muchos-a-muchos} entre las clases \textit{Group} y \textit{Role}.

\lstinputlisting[language=Java,caption={C�digo fuente de la clase Group.java},label=code:Group.java]{..//..//desglosa-web//src//main//java//es//uclm//inf_cr//alarcos//desglosa_web//model//Group.java}



\subsubsection{Clase Role.java}
\label{anexo:role}

En la implementaci�n de esta clase destacan sus anotaciones mediante JPA (Java Persistence API) y que implementa el \textit{interface} \textbf{GrantedAuthority} provisto por Spring Security.

\lstinputlisting[language=Java,caption={C�digo fuente de la clase Role.java},label=code:Role.java]{..//..//desglosa-web//src//main//java//es//uclm//inf_cr//alarcos//desglosa_web//model//Role.java}



\subsubsection{Clase CustomUserDetailsService.java}
\label{anexo:customuserdetailsservice}

Esta clase implementa \textbf{UserDetailsService}, provisto por Spring Security, y extiende de HibernateDaoSupport, provisto por Spring. Es la clase responsable de indicar si el usuario se autentica correctamente y, en caso afirmativo, otorgarle los grupos y roles que le correspondan.

\lstinputlisting[language=Java,caption={C�digo fuente de la clase CustomUserDetailsService.java},label=code:CustomUserDetailsService.java]{..//..//desglosa-web//src//main//java//es//uclm//inf_cr//alarcos//desglosa_web//security//CustomUserDetailsService.java}



\subsection{Configuraci�n de Spring Security}
\label{anexo:applicationcontext}

La configuraci�n de Spring Security se lleva a cabo mediante el siguiente fichero XML. En �l se indica qu� recursos y p�ginas son accesibles por cada rol, as� como el comportamiento de la aplicaci�n web ante las distintas operaciones de autenticaci�n, cierre de sesi�n o accesos denegados.

\lstinputlisting[language=XML,caption={Fichero de configuraci�n de Spring Security: applicationContext-security.xml},label=code:applicationcontext-security]{..//..//desglosa-web//src//main//resources//applicationContext-security.xml}





\section{Creaci�n de un perfil para ejecutar pruebas funcionales de una aplicaci�n web mediante Maven}

\lstinputlisting[language=XML,caption={Snippet para la automatizaci�n de pruebas funcionales de la aplicaci�n web mediante Maven},label=anexo:integration-test-profile]{snippets/integration-test-profile.txt}