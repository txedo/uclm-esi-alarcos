\chapter{Estado del Arte}
\label{ch:estadodelarte}

% MOISES

% En esta parte se deben mostrar los conocimientos obtenidos en la b�squeda bibliogr�fica y no ideas personales del autor; como mucho podr� aportar un comentario de algunas pocas ideas extra�das de las fuentes en que se ha basado.

% Se articular� esta parte en diversas secciones, que permitan la exposici�n estructurada y did�ctica de los conocimientos de la investigaci�n bibliogr�fica.

%\section{La calidad del software}
%\section{Modelos de calidad}
%\section{Visualizaci�n de la calidad del software}
%\section{Librer�as de visualizaci�n}
%\section{Herramientas existentes}








% MARIAN

% Globalizaci�n
% Calidad y medidas
% Visualizaci�n
% Perspectiva de en qu� cnosiste y qu� hay en la literatura �CUIDADO REFERENCIAS!
% Herramientas relacionadas (si hay tiempo, secundario)
