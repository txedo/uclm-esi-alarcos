\chapter{Estado del Arte}
\label{ch:estadodelarte}

% En esta parte se deben mostrar los conocimientos obtenidos en la b�squeda bibliogr�fica y no ideas personales del autor; como mucho podr� aportar un comentario de algunas pocas ideas extra�das de las fuentes en que se ha basado.

% Se articular� esta parte en diversas secciones, que permitan la exposici�n estructurada y did�ctica de los conocimientos de la investigaci�n bibliogr�fica.

% Globalizaci�n
% Calidad y medidas software (nombrar modelos de calidad?)
% Visualizaci�n (de la calidad del software)
% Perspectiva de en qu� consiste y qu� hay en la literatura �CUIDADO REFERENCIAS!
% Librer�as de visualizaci�n
% Herramientas relacionadas (si hay tiempo, secundario)




%Ventajas: 
%* Obtener la mano de obra, en ocasiones escasa, necesaria para el proyecto en pa�ses donde se est� avanzando notablemente en I+D.
%* Disminuir las dificultades para construir los m�todos necesarios para gestionar el conocimiento y coordinar los distintos grupos de trabajo, abstray�ndose de la distancia geogr�fica \cite{kobitzsch01}.
%* Recortar costes y gastos derivados de los viajes a los pa�ses con los que se colabora.