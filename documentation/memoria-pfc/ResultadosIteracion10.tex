En esta �ltima iteraci�n del ciclo se han corregido algunos \textit{bugs} que se han detectado a lo largo del ciclo de desarrollo. Adem�s, tal y como se indica en la ficha de iteraci�n \ref{table:ficha-iteracion-10} se elaboran los documentos asociados al proyecto:
\begin{itemize}
	\item \textbf{Manual de usuario}. Documento que explica las principales funcionalidades de las que dispone el sistema desarrollado (V�ase Anexo \ref{ch:manualdeusuario}).
	\item \textbf{Manual de instalaci�n y configuraci�n del entorno de desarrollo}. Documento que ser� vital para el equipo que se encargue del mantenimiento del sistema (V�ase Anexo \ref{ch:manualentorno}).
\end{itemize}

A continuaci�n se procede la creaci�n del paquete de distribuci�n y su instalaci�n en la empresa. Este proceso consta de dos partes, una para cada componente:
\begin{itemize}
	\item \textbf{Motor gr�fico}. Este componente ser� distribuido en un paquete JAR (Java ARchive). Gracias al uso de Maven para la gesti�n del ciclo de vida del proyecto, basta con ejecutar el comando \textit{mvn package} para obtener el correspondiente archivo. A continuaci�n, se colocar� el fichero JAR obtenido bajo el directorio \textit{src\\main\\webapp\\applet} con el nombre \textit{desglosa.jar}.
	\item \textbf{Aplicaci�n web}. Este componente ser� distribuido en formato WAR (Web application ARchive), que contiene todo el c�digo ejecutable de la aplicaci�n web. Una vez m�s, se ejecutar� el comando \textit{mvn package} para obtener el fichero correspondiente, que deber� ser desplegado en un contenedor de servlets y p�ginas JSP.
\end{itemize}

Para finalizar, cabe destacar la posibilidad de la generaci�n autom�tica de p�ginas web para la distribuci�n de cada uno de los componentes que conforman este PFC. Esto es posible, una vez m�s, gracias al uso de Maven. Para generar dichas p�ginas web bastar� con ejecutar el comando \textit{mvn site} para cada uno de los componentes.
