\section{M�todo y fases de trabajo}

Dada la naturaleza del proyecto, en el que se ir�n definiendo requisitos y funcionalidades a lo largo de todo su ciclo de vida, se ha optado por utilizar una metodolog�a de desarrollo de software que permita adaptarse a este caso de estudio. Por estas razones, se ha seleccionado el \textbf{Proceso Unificado de Desarrollo} (en adelante PUD) como metodolog�a de trabajo.

El PUD \cite{jac00} es una evoluci�n del Proceso Unificado de Rational, que define un \textit{``conjunto de actividades necesarias para transformar los requisitos de usuario en un sistema software''}. Es un marco gen�rico que puede especializarse para una gran variedad de sistemas de software cualesquiera que sean el �rea de aplicaci�n o el tama�o del proyecto \cite{pud-espinoza}. Sus principales caracter�sticas se abordan en las siguientes subsecciones.


\subsection{Dirigido por casos de uso}

Para poder desarrollar un sistema es necesario saber qu� necesitan sus usuarios \cite{pud-espinoza}. Un usuario puede ser un ser humano u otro sistema que interacciona con el sistema que se est� desarrollando. Las necesidades de un usuario se denominan requisitos funcionales y se representan por medio de casos de uso.
	
Los casos de uso gu�an el proceso de desarrollo desde la especificaci�n de requisitos hasta las pruebas y se utilizan para crear modelos que permitan la construcci�n e implementaci�n de los mismos. Todos los casos de uso juntos constituyen el \textbf{modelo de casos de uso} \cite{pud-torossi}.


\subsection{Centrado en la arquitectura}

Un sistema software puede contemplarse desde varios puntos de vista. La arquitectura software incluye los aspectos est�ticos y din�micos m�s significativos del sistema, y debe estar profundamente relacionada con los casos de uso ya que debe permitir el desarrollo de los mismos \cite{pud-torossi}. Por esta raz�n, la arquitectura y los casos de uso deben desarrollarse en paralelo.


\subsection{Iterativo e incremental}

Partiendo del dicho latino \textit{divide et impera} (en castellano, \textit{divide y vencer�s}), el ciclo de vida del proyecto se divide en ciclos. Cada ciclo consta de cuatro fases (inicio, elaboraci�n, construcci�n y transici�n) que a su vez se dividen en iteraciones (Fig. \ref{fig:mapa-conceptual-pud}).

\imagenBorde{./images/mapa-conceptual-pud.png}{0.50}{Mapa conceptual del PUD}{fig:mapa-conceptual-pud}

	 Cada iteraci�n da resultado a una nueva versi�n del proyecto, aborda un conjunto de casos de uso que ampl�an la funcionalidad global del producto desarrollado y trata los riesgos m�s importantes. Si la iteraci�n cumple sus objetivos se contin�a con la pr�xima, en caso contrario se vuelve a las primeras etapas de la misma y se toman otras decisiones que permitan cumplir sus objetivos. Las iteraciones deben ser seleccionadas y ejecutarse de forma controlada y siguiendo un flujo de trabajo \cite{pud-torossi}. Este flujo suele constar de las t�picas etapas de captaci�n de requisitos, an�lisis, dise�o, implementaci�n, pruebas y despliegue.
