\section{M�todo y fases de trabajo}

Dada la naturaleza del proyecto, en el que se ir�n definiendo requisitos y funcionalidades a lo largo de todo su ciclo de vida, se ha optado por utilizar una metodolog�a de desarrollo de software que permita adaptarse a este caso de estudio. Por estas razones, se ha seleccionado el \textbf{Proceso Unificado de Desarrollo} (en adelante PUD) como metodolog�a de trabajo.

El PUD \cite{jac00} es una evoluci�n del Proceso Unificado de Rational, que define un \textit{``conjunto de actividades necesarias para transformar los requisitos de usuario en un sistema software''}. Es un marco gen�rico que puede especializarse para una gran variedad de sistemas de software cualesquiera que sean el �rea de aplicaci�n o el tama�o del proyecto \cite{pud-espinoza}. Sus caracteriza debido a que est� dirigido por casos de uso, centrado en la arquitectura, y es iterativo e incremental (Fig. \ref{fig:mapa-conceptual-pud}).

\imagenBorde{./images/mapa-conceptual-pud.png}{0.50}{Mapa conceptual del PUD}{fig:mapa-conceptual-pud}
