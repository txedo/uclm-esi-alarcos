\section{M�todo y fases de trabajo}

Dada la naturaleza del proyecto, en el que se ir�n definiendo requisitos y funcionalidades a lo largo de todo su ciclo de vida, se ha optado por utilizar una metodolog�a de desarrollo de software que permita adaptarse a este caso de estudio. Por estas razones, se ha seleccionado el \textbf{Proceso Unificado de Desarrollo} (en adelante PUD) como metodolog�a de trabajo.

El PUD \cite{jac00} define un \textit{``conjunto de actividades necesarias para transformar los requisitos de usuario en un sistema software''}. Es un marco gen�rico que puede especializarse para una gran variedad de sistemas de software cualesquiera que sean el �rea de aplicaci�n o el tama�o del proyecto \cite{pud-espinoza}. Sus caracter�sticas principales son que est� dirigido por casos de uso, centrado en la arquitectura, y es iterativo e incremental.

Para llevar esta metodolog�a a la pr�ctica se establecer�n una serie de ciclos, dentro de los cuales se definir� un conjunto de requisitos funcionales y no funcionales a los que habr� que dar soluci�n. Los requisitos negociados dentro de cada ciclo conforman los hitos que deber�n alcanzarse en dicho ciclo. Los ciclos ser�n divididos a su vez en iteraciones en las que se abordar�n un subconjunto de requisitos. Para concluir una iteraci�n ser� necesario planificar reuniones con los responsables del proyecto de \textbf{Indra Software Labs S.L.}, que deber�n validar y verificar que se han alcanzado los hitos establecidos para dicha iteraci�n. Si la validaci�n y verificaci�n es de car�cter positivo, se obtendr� una nueva versi�n del software. Este proceso se aplicar� hasta obtener la versi�n final del software  (Fig. \ref{fig:mapa-conceptual-pud}).

\imagenBorde{./images/mapa-conceptual-pud.png}{0.50}{Mapa conceptual del PUD}{fig:mapa-conceptual-pud}