\section{Introducci�n}

La \textbf{globalizaci�n}, seg�n la Real Academia Espa�ola, es la \textit{``tendencia de los mercados y de las empresas a extenderse, alcanzando una dimensi�n mundial que sobrepasa las fronteras nacionales''}. Otras fuentes \cite{croucher04} buscan una definici�n m�s expl�cita y definen este concepto como \textit{``un proceso econ�mico, tecnol�gico, social, pol�tico y cultural de interacci�n e integraci�n entre personas, empresas y gobiernos de todo el mundo''}. No cabe duda de que este proceso o tendencia ha desencadenado una gran cantidad de cambios no s�lo en la forma de trabajo tradicional de la industria y el comercio, si no tambi�n en el modo en el que los productos son dise�ados, construidos, probados y distribuidos \cite{ohara94}. Por lo tanto, la industria del software y las empresas que dependen del software tambi�n se ven afectados por este movimiento y sus consecuencias.

Debido al fen�meno de la globalizaci�n, el software est� siendo desarrollado por personas que no se encuentran co-localizadas en las mismas ubicaciones, es decir, se encuentran distribuidas o dispersas geogr�ficamente. Esta distribuci�n geogr�fica puede limitarse a un �rea cerrada y pr�xima (por ejemplo, distintos edificios en una misma localidad) o expandirse por todo el mundo evadiendo cualquier frontera. Adem�s, los integrantes de un equipo de desarrollo pueden pertenecer a una misma organizaci�n o a organizaciones diferentes \cite{katzy99}. Como consecuencia de estas dos caracter�sticas surge un nuevo paradigma denominado \textbf{Desarrollo Global de Software} (en ingl�s, Global Software Development o GSD).

Este nuevo paradigma introduce una serie de caracter�sticas y complejidades que puede afectar tanto al desarrollo como a la gesti�n de proyectos. Es de vital importancia que las empresas manejen m�tricas e indicadores que les permitan comprobar si determinados factores est�n siendo llevados a cabo con normalidad o que reflejen si existe alg�n tipo de riesgo o anomal�a. Para ello se propone este Proyecto Fin de Carrera, en adelante PFC, que tratar� de dar una soluci�n para la definici�n de dichas m�tricas y dotarlas de una representaci�n visual intuitiva, eficaz y eficiente.
